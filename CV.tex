%!TEX TS-program = lualatex
%!TEX encoding   = UTF-8 Unicode

% YSS NOTE
% Git Usage
% when in local get the latest on overleaf
%   git pull overleaf master --allow-unrelated-histories --rebase=false
% when in overleaf get the latest from github
%   git push overleaf master --set-upstream
% if pushing a different branch to overleaf master
%   git push overleaf branch-name:master


% This is the simplest class to start with. You may want to change the
% paper format or the standard font size.
\documentclass[letter,11pt]{article} %\documentclass[a4paper,11pt]{article}


\usepackage[british]{babel}       % Change this 
\usepackage         {fontspec}    % Font selection
\usepackage         {fontawesome} % Symbols/icons

% Change this depending on your own preferences. I personally prefer
% nice ligatures and 'old style' numbers. Additionally, the way each
% font is set up ensures that the layout is very uniform---mostly, I
% achieve this effect using `MatchLowercase`.
%
% TODO: Make sure you select a font with small caps here, unless you
% also change the section style. See below.
\defaultfontfeatures{Ligatures=TeX}
\setmainfont[Renderer=Basic,Numbers={Proportional,OldStyle}]{MinionPro-Regular.otf}
\setsansfont[Scale=MatchLowercase]{Cabin Regular}
\setmonofont[Scale=MatchLowercase]{Fira Mono}

\setlength{\parindent}   {0pt}  % No indentation
\setlength{\marginparsep}{10pt} % Separation of margin notes

%%%%%%%%%%%%%%%%%%%%%%%%%%%%%%%%%%%%%%%%%%%%%%%%%%%%%%%%%%%%%%%%%%%%%%%%
% Document layout
%%%%%%%%%%%%%%%%%%%%%%%%%%%%%%%%%%%%%%%%%%%%%%%%%%%%%%%%%%%%%%%%%%%%%%%%

% Permits a very precise adjustment of all the margins. Since a CV is
% a document where *you* should exhibit *your* preferences, feel free
% to adjust this the way you want.
%
% I am not aware of any good rules here.
\usepackage{geometry}
\geometry{
  a4paper,
  textwidth      = 14.0cm,
  textheight     = 25.0cm,
  marginparwidth =  2.5cm
}

\setlength{\parindent}{0pt}     % No indentations for paragraphs
\setlength{\skip\footins}{2cm}  % Reduced footer distance

% A nice way to typeset proper ordinal superscripts
\newcommand  {\rd}{\textsuperscript{\textup{rd}}\xspace}
\newcommand  {\nd}{\textsuperscript{\textup{nd}}\xspace}
\renewcommand{\th}{\textsuperscript{\textup{th}}\xspace}

% To be used to indicate equal contributions of two or more authors in
% a paper.
\newcommand{\authorequal}{\kern-0.1em\textsuperscript{\dagger}}

%%%%%%%%%%%%%%%%%%%%%%%%%%%%%%%%%%%%%%%%%%%%%%%%%%%%%%%%%%%%%%%%%%%%%%%%
% Margins macro
%%%%%%%%%%%%%%%%%%%%%%%%%%%%%%%%%%%%%%%%%%%%%%%%%%%%%%%%%%%%%%%%%%%%%%%%
%
% This permits you to put some comments into the margins of the
% document; I borrowed this idea from Tufte's works, but in the
% CV, I only use it to indicate durations.

\usepackage{marginnote}
\renewcommand*{\raggedleftmarginnote}{}
\newcommand{\years}[1]{%
  % Forces notes to appear on the left
  {\reversemarginpar\strut\marginnote{{\small#1}}}%
}


% Required to support footnote references; has to come *after* the
% `marginnote` package, though, because it redefines some things.
\usepackage{scrextend}

%%%%%%%%%%%%%%%%%%%%%%%%%%%%%%%%%%%%%%%%%%%%%%%%%%%%%%%%%%%%%%%%%%%%%%%%
% Section style
%%%%%%%%%%%%%%%%%%%%%%%%%%%%%%%%%%%%%%%%%%%%%%%%%%%%%%%%%%%%%%%%%%%%%%%%

% This styles *all* the sections exactly the same: regular font, with
% small caps.
%
% TODO: Adjust this if you change the font---otherwise, everything will
% look super strange.
\usepackage{sectsty}
\allsectionsfont{\mdseries\scshape}

% Typographical adjustments. If you have longer blocks of texts, such as
% publications, in your CV, this will give the text a more natural look.
%
% TODO: Some people do not like this. If you are among them, just change
% this or remove it
\usepackage{microtype}

%%%%%%%%%%%%%%%%%%%%%%%%%%%%%%%%%%%%%%%%%%%%%%%%%%%%%%%%%%%%%%%%%%%%%%%%
% Colours
%%%%%%%%%%%%%%%%%%%%%%%%%%%%%%%%%%%%%%%%%%%%%%%%%%%%%%%%%%%%%%%%%%%%%%%%

% TODO: redefine this colour if you do not like it.
\usepackage[usenames,dvipsnames]{color}
\definecolor{cardinal}{RGB}{76, 0, 153}

%%%%%%%%%%%%%%%%%%%%%%%%%%%%%%%%%%%%%%%%%%%%%%%%%%%%%%%%%%%%%%%%%%%%%%%%
% PDF setup
%%%%%%%%%%%%%%%%%%%%%%%%%%%%%%%%%%%%%%%%%%%%%%%%%%%%%%%%%%%%%%%%%%%%%%%%

\usepackage[
  bookmarks,  % Create PDF bookmarks
  colorlinks, % Use the link colour from above
  breaklinks  % Break links in the text
]{hyperref}

\hypersetup{%
  urlcolor=cardinal,
  pdfauthor={Yasaman S. Sefidgar},                  % TODO: Add your own name here
  pdftitle={Yasaman S. Sefidgar: Curriculum vitae}, % TODO: Add your own name here
  pdfproducer={}
}

%%%%%%%%%%%%%%%%%%%%%%%%%%%%%%%%%%%%%%%%%%%%%%%%%%%%%%%%%%%%%%%%%%%%%%%%
% Penalties
%%%%%%%%%%%%%%%%%%%%%%%%%%%%%%%%%%%%%%%%%%%%%%%%%%%%%%%%%%%%%%%%%%%%%%%%

\clubpenalty         = 10000
\displaywidowpenalty = 10000
\widowpenalty        = 10000

\usepackage[defaultlines=100,all]{nowidow} % rejects all orphaned and widowed lines

%%%%%%%%%%%%%%%%%%%%%%%%%%%%%%%%%%%%%%%%%%%%%%%%%%%%%%%%%%%%%%%%%%%%%%%%
% Main document
%%%%%%%%%%%%%%%%%%%%%%%%%%%%%%%%%%%%%%%%%%%%%%%%%%%%%%%%%%%%%%%%%%%%%%%%

\begin{document}
{
  \Huge \textsc{Yasaman S. Sefidgar}
  \begin{flushleft}
    \scriptsize
      \faPhone\space+1 (206) 660-0795
      \enspace\faEnvelope\space{\href{mailto:einsian@cs.washington.edu}{einsian@cs.washington.edu}}
      \enspace\faHome\space{\href{https://homes.cs.washington.edu/~einsian/}{homes.cs.washington.edu/{\textasciitilde}einsian}}
  \end{flushleft}
}

%%%%%%%%%%%%%%%%%%%%%%%%%%%%%%%%%%%%%%%%%%%%%%%%%%%%%%%%%%%%%%%%%%%%%%%%
\section*{Summary}
%%%%%%%%%%%%%%%%%%%%%%%%%%%%%%%%%%%%%%%%%%%%%%%%%%%%%%%%%%%%%%%%%%%%%%%%
This is a \textbf{test}

%%%%%%%%%%%%%%%%%%%%%%%%%%%%%%%%%%%%%%%%%%%%%%%%%%%%%%%%%%%%%%%%%%%%%%%%
\section*{Research interests}
%%%%%%%%%%%%%%%%%%%%%%%%%%%%%%%%%%%%%%%%%%%%%%%%%%%%%%%%%%%%%%%%%%%%%%%%

Cosmology \textbullet{} Astrophysics \textbullet{} Mathematics

%%%%%%%%%%%%%%%%%%%%%%%%%%%%%%%%%%%%%%%%%%%%%%%%%%%%%%%%%%%%%%%%%%%%%%%%
\section*{Education}
%%%%%%%%%%%%%%%%%%%%%%%%%%%%%%%%%%%%%%%%%%%%%%%%%%%%%%%%%%%%%%%%%%%%%%%%

\years{1927}%
\textbf{Ph.D.}\ in physics at MIT\\[.1cm]
\emph{The gravitational field in a fluid sphere of uniform invariant density according to the theory of relativity}\\[.05cm]
Advisor: Arthur Eddington\\
\years{1920}%
\textbf{D.Sc.}\ in mathematics at Catholic University of Leuven\\[.1cm]
\emph{l'Approximation des fonctions de plusieurs variables réelles}\\[.05cm]
Advisor: Charles de la Vallée-Poussin\\[0.2cm]
\years{1911--1914}%
Studies of civil engineering at Catholic University of Leuven

%%%%%%%%%%%%%%%%%%%%%%%%%%%%%%%%%%%%%%%%%%%%%%%%%%%%%%%%%%%%%%%%%%%%%%%%
\section*{Publications}
%%%%%%%%%%%%%%%%%%%%%%%%%%%%%%%%%%%%%%%%%%%%%%%%%%%%%%%%%%%%%%%%%%%%%%%%

\years{1946}%
%
G.\ Lemaître, The Primeval Atom --- An Essay on Cosmogony\\[.2cm]
%
\years{1931}%
G.\ Lemaître, L'Hypothèse de l'atome primitif\\[.2cm]
%
\years{1927}%
%
G.\ Lemaître, \emph{Un univers homogène de masse constante et de rayon
croissant rendant compte de la vitesse radiale des nébuleuses
extragalactiques}, Annals of the Scientific Society of Brussels, 47A:41.\\[.2cm]
%
G.\ Lemaître, Discussion sur l'évolution de l'univers

%%%%%%%%%%%%%%%%%%%%%%%%%%%%%%%%%%%%%%%%%%%%%%%%%%%%%%%%%%%%%%%%%%%%%%%%
\section*{Honours}
%%%%%%%%%%%%%%%%%%%%%%%%%%%%%%%%%%%%%%%%%%%%%%%%%%%%%%%%%%%%%%%%%%%%%%%%

\years{1953}%
Received the inaugural \emph{Eddington Medal}, awarded
by the Royal Astronomical Society\\[.2cm]
%
\years{1950}%
%
Received the decennial prize for applied sciences for the period
1933--1942 by the Belgian government\\[.2cm]
%
\years{1936}%
Received the \emph{Prix Jules Janssen}, the highest
award of the Société astronomique de France\footnote{The French
astronomical society}\\[.2cm]
%
\years{1934}%
Received the \emph{Francqui Prize}, the highest Belgian
scientific distinction, from King Leopold III, proposed by Albert
Einstein, Charles de la Vallée-Poussin, and Alexandre de Hemptinne\\[.2cm]
%
Received the \emph{Mendel Medal} of Villanova University.\\[.2cm]

%%%%%%%%%%%%%%%%%%%%%%%%%%%%%%%%%%%%%%%%%%%%%%%%%%%%%%%%%%%%%%%%%%%%%%%%
\section*{About this template}
%%%%%%%%%%%%%%%%%%%%%%%%%%%%%%%%%%%%%%%%%%%%%%%%%%%%%%%%%%%%%%%%%%%%%%%%

This document is modelled after the style of
\href{https://bastian.rieck.me}{my} own CV, which you can find
\href{https://bastian.rieck.me/about/cv.pdf}{here}. It was originally
inspired by the CV of \href{http://nitens.org/taraborelli/cvtex}{Dario
Taraborelli}, but has since somewhat evolved---or so I like to tell
myself---from the original template.\\

This document does not offer many special features, except for the
\verb|\years| macro, which can be used to typeset small notes in the
margin of the document. I use them to indicate \emph{durations}, but you
could also repurpose it to have small annotations in the style of Edward
Tufte. In addition, the template uses two special fonts for sans-serif
typesetting and monospace typesetting. I do not tend to use the former
for many things, but you might like it for typesetting the titles of
publications. The latter font type, though, I often use in order to
describe software projects or packages, such as \texttt{PyTorch} or
\texttt{scikit-learn}.\\

That's all there is to it---enjoy the template \& feel free to open
tickets for comments or feedback.

\end{document}
